%\documentclass{article}
%%\usepackage[html,png]{tex4ht}
%\usepackage{hyperref}
%
%\setlength\parindent{0pt}
%
%\title{Nanotubes}
%\date{}
%\begin{document}
%\maketitle

\section{Nanotubes}

Although NanoCap was designed to produce fullerenes and cap nanotubes, isolated nanotubes can be constructed. These can be either finite 
in length or periodic along the axial direction.

A nanotube is constructed via:

\textit{File--$>$New Structure--$>$Single Structure}

them click \textit{Nanotube} from the popup menu.

\subsection{Finite tubes}

Finite tubes are simply constructed as close to the user defined length as possible. 
This length is defined in the \textbf{Calculations--$>$Input} options.
 
\subsection{Periodic tubes}

Periodic tubes are constructed using a user defined number of unit cells in the \textit{z} direction. 
The number of unit cells can also be found in the \textbf{Calculations--$>$Input} options.

The nanotube will be associated with a \textit{periodic length} which can be found in the
\textbf{Information} options tab. This will be required by simulation software if the nanotube is to be
used in a periodic simulation.

%\end{document}