% Generated by Sphinx.
\def\sphinxdocclass{report}
\documentclass[letterpaper,10pt,english]{sphinxmanual}
\usepackage[utf8]{inputenc}
\DeclareUnicodeCharacter{00A0}{\nobreakspace}
\usepackage[T1]{fontenc}
\usepackage{babel}
\usepackage{times}
\usepackage[Bjarne]{fncychap}
\usepackage{longtable}
\usepackage{sphinx}
\usepackage{multirow}


\title{pybdf Documentation}
\date{July 08, 2013}
\release{('0.1.15',)}
\author{Samuele Carcagno}
\newcommand{\sphinxlogo}{}
\renewcommand{\releasename}{Release}
\makeindex

\makeatletter
\def\PYG@reset{\let\PYG@it=\relax \let\PYG@bf=\relax%
    \let\PYG@ul=\relax \let\PYG@tc=\relax%
    \let\PYG@bc=\relax \let\PYG@ff=\relax}
\def\PYG@tok#1{\csname PYG@tok@#1\endcsname}
\def\PYG@toks#1+{\ifx\relax#1\empty\else%
    \PYG@tok{#1}\expandafter\PYG@toks\fi}
\def\PYG@do#1{\PYG@bc{\PYG@tc{\PYG@ul{%
    \PYG@it{\PYG@bf{\PYG@ff{#1}}}}}}}
\def\PYG#1#2{\PYG@reset\PYG@toks#1+\relax+\PYG@do{#2}}

\expandafter\def\csname PYG@tok@gd\endcsname{\def\PYG@tc##1{\textcolor[rgb]{0.63,0.00,0.00}{##1}}}
\expandafter\def\csname PYG@tok@gu\endcsname{\let\PYG@bf=\textbf\def\PYG@tc##1{\textcolor[rgb]{0.50,0.00,0.50}{##1}}}
\expandafter\def\csname PYG@tok@gt\endcsname{\def\PYG@tc##1{\textcolor[rgb]{0.00,0.25,0.82}{##1}}}
\expandafter\def\csname PYG@tok@gs\endcsname{\let\PYG@bf=\textbf}
\expandafter\def\csname PYG@tok@gr\endcsname{\def\PYG@tc##1{\textcolor[rgb]{1.00,0.00,0.00}{##1}}}
\expandafter\def\csname PYG@tok@cm\endcsname{\let\PYG@it=\textit\def\PYG@tc##1{\textcolor[rgb]{0.25,0.50,0.56}{##1}}}
\expandafter\def\csname PYG@tok@vg\endcsname{\def\PYG@tc##1{\textcolor[rgb]{0.73,0.38,0.84}{##1}}}
\expandafter\def\csname PYG@tok@m\endcsname{\def\PYG@tc##1{\textcolor[rgb]{0.13,0.50,0.31}{##1}}}
\expandafter\def\csname PYG@tok@mh\endcsname{\def\PYG@tc##1{\textcolor[rgb]{0.13,0.50,0.31}{##1}}}
\expandafter\def\csname PYG@tok@cs\endcsname{\def\PYG@tc##1{\textcolor[rgb]{0.25,0.50,0.56}{##1}}\def\PYG@bc##1{\setlength{\fboxsep}{0pt}\colorbox[rgb]{1.00,0.94,0.94}{\strut ##1}}}
\expandafter\def\csname PYG@tok@ge\endcsname{\let\PYG@it=\textit}
\expandafter\def\csname PYG@tok@vc\endcsname{\def\PYG@tc##1{\textcolor[rgb]{0.73,0.38,0.84}{##1}}}
\expandafter\def\csname PYG@tok@il\endcsname{\def\PYG@tc##1{\textcolor[rgb]{0.13,0.50,0.31}{##1}}}
\expandafter\def\csname PYG@tok@go\endcsname{\def\PYG@tc##1{\textcolor[rgb]{0.19,0.19,0.19}{##1}}}
\expandafter\def\csname PYG@tok@cp\endcsname{\def\PYG@tc##1{\textcolor[rgb]{0.00,0.44,0.13}{##1}}}
\expandafter\def\csname PYG@tok@gi\endcsname{\def\PYG@tc##1{\textcolor[rgb]{0.00,0.63,0.00}{##1}}}
\expandafter\def\csname PYG@tok@gh\endcsname{\let\PYG@bf=\textbf\def\PYG@tc##1{\textcolor[rgb]{0.00,0.00,0.50}{##1}}}
\expandafter\def\csname PYG@tok@ni\endcsname{\let\PYG@bf=\textbf\def\PYG@tc##1{\textcolor[rgb]{0.84,0.33,0.22}{##1}}}
\expandafter\def\csname PYG@tok@nl\endcsname{\let\PYG@bf=\textbf\def\PYG@tc##1{\textcolor[rgb]{0.00,0.13,0.44}{##1}}}
\expandafter\def\csname PYG@tok@nn\endcsname{\let\PYG@bf=\textbf\def\PYG@tc##1{\textcolor[rgb]{0.05,0.52,0.71}{##1}}}
\expandafter\def\csname PYG@tok@no\endcsname{\def\PYG@tc##1{\textcolor[rgb]{0.38,0.68,0.84}{##1}}}
\expandafter\def\csname PYG@tok@na\endcsname{\def\PYG@tc##1{\textcolor[rgb]{0.25,0.44,0.63}{##1}}}
\expandafter\def\csname PYG@tok@nb\endcsname{\def\PYG@tc##1{\textcolor[rgb]{0.00,0.44,0.13}{##1}}}
\expandafter\def\csname PYG@tok@nc\endcsname{\let\PYG@bf=\textbf\def\PYG@tc##1{\textcolor[rgb]{0.05,0.52,0.71}{##1}}}
\expandafter\def\csname PYG@tok@nd\endcsname{\let\PYG@bf=\textbf\def\PYG@tc##1{\textcolor[rgb]{0.33,0.33,0.33}{##1}}}
\expandafter\def\csname PYG@tok@ne\endcsname{\def\PYG@tc##1{\textcolor[rgb]{0.00,0.44,0.13}{##1}}}
\expandafter\def\csname PYG@tok@nf\endcsname{\def\PYG@tc##1{\textcolor[rgb]{0.02,0.16,0.49}{##1}}}
\expandafter\def\csname PYG@tok@si\endcsname{\let\PYG@it=\textit\def\PYG@tc##1{\textcolor[rgb]{0.44,0.63,0.82}{##1}}}
\expandafter\def\csname PYG@tok@s2\endcsname{\def\PYG@tc##1{\textcolor[rgb]{0.25,0.44,0.63}{##1}}}
\expandafter\def\csname PYG@tok@vi\endcsname{\def\PYG@tc##1{\textcolor[rgb]{0.73,0.38,0.84}{##1}}}
\expandafter\def\csname PYG@tok@nt\endcsname{\let\PYG@bf=\textbf\def\PYG@tc##1{\textcolor[rgb]{0.02,0.16,0.45}{##1}}}
\expandafter\def\csname PYG@tok@nv\endcsname{\def\PYG@tc##1{\textcolor[rgb]{0.73,0.38,0.84}{##1}}}
\expandafter\def\csname PYG@tok@s1\endcsname{\def\PYG@tc##1{\textcolor[rgb]{0.25,0.44,0.63}{##1}}}
\expandafter\def\csname PYG@tok@gp\endcsname{\let\PYG@bf=\textbf\def\PYG@tc##1{\textcolor[rgb]{0.78,0.36,0.04}{##1}}}
\expandafter\def\csname PYG@tok@sh\endcsname{\def\PYG@tc##1{\textcolor[rgb]{0.25,0.44,0.63}{##1}}}
\expandafter\def\csname PYG@tok@ow\endcsname{\let\PYG@bf=\textbf\def\PYG@tc##1{\textcolor[rgb]{0.00,0.44,0.13}{##1}}}
\expandafter\def\csname PYG@tok@sx\endcsname{\def\PYG@tc##1{\textcolor[rgb]{0.78,0.36,0.04}{##1}}}
\expandafter\def\csname PYG@tok@bp\endcsname{\def\PYG@tc##1{\textcolor[rgb]{0.00,0.44,0.13}{##1}}}
\expandafter\def\csname PYG@tok@c1\endcsname{\let\PYG@it=\textit\def\PYG@tc##1{\textcolor[rgb]{0.25,0.50,0.56}{##1}}}
\expandafter\def\csname PYG@tok@kc\endcsname{\let\PYG@bf=\textbf\def\PYG@tc##1{\textcolor[rgb]{0.00,0.44,0.13}{##1}}}
\expandafter\def\csname PYG@tok@c\endcsname{\let\PYG@it=\textit\def\PYG@tc##1{\textcolor[rgb]{0.25,0.50,0.56}{##1}}}
\expandafter\def\csname PYG@tok@mf\endcsname{\def\PYG@tc##1{\textcolor[rgb]{0.13,0.50,0.31}{##1}}}
\expandafter\def\csname PYG@tok@err\endcsname{\def\PYG@bc##1{\setlength{\fboxsep}{0pt}\fcolorbox[rgb]{1.00,0.00,0.00}{1,1,1}{\strut ##1}}}
\expandafter\def\csname PYG@tok@kd\endcsname{\let\PYG@bf=\textbf\def\PYG@tc##1{\textcolor[rgb]{0.00,0.44,0.13}{##1}}}
\expandafter\def\csname PYG@tok@ss\endcsname{\def\PYG@tc##1{\textcolor[rgb]{0.32,0.47,0.09}{##1}}}
\expandafter\def\csname PYG@tok@sr\endcsname{\def\PYG@tc##1{\textcolor[rgb]{0.14,0.33,0.53}{##1}}}
\expandafter\def\csname PYG@tok@mo\endcsname{\def\PYG@tc##1{\textcolor[rgb]{0.13,0.50,0.31}{##1}}}
\expandafter\def\csname PYG@tok@mi\endcsname{\def\PYG@tc##1{\textcolor[rgb]{0.13,0.50,0.31}{##1}}}
\expandafter\def\csname PYG@tok@kn\endcsname{\let\PYG@bf=\textbf\def\PYG@tc##1{\textcolor[rgb]{0.00,0.44,0.13}{##1}}}
\expandafter\def\csname PYG@tok@o\endcsname{\def\PYG@tc##1{\textcolor[rgb]{0.40,0.40,0.40}{##1}}}
\expandafter\def\csname PYG@tok@kr\endcsname{\let\PYG@bf=\textbf\def\PYG@tc##1{\textcolor[rgb]{0.00,0.44,0.13}{##1}}}
\expandafter\def\csname PYG@tok@s\endcsname{\def\PYG@tc##1{\textcolor[rgb]{0.25,0.44,0.63}{##1}}}
\expandafter\def\csname PYG@tok@kp\endcsname{\def\PYG@tc##1{\textcolor[rgb]{0.00,0.44,0.13}{##1}}}
\expandafter\def\csname PYG@tok@w\endcsname{\def\PYG@tc##1{\textcolor[rgb]{0.73,0.73,0.73}{##1}}}
\expandafter\def\csname PYG@tok@kt\endcsname{\def\PYG@tc##1{\textcolor[rgb]{0.56,0.13,0.00}{##1}}}
\expandafter\def\csname PYG@tok@sc\endcsname{\def\PYG@tc##1{\textcolor[rgb]{0.25,0.44,0.63}{##1}}}
\expandafter\def\csname PYG@tok@sb\endcsname{\def\PYG@tc##1{\textcolor[rgb]{0.25,0.44,0.63}{##1}}}
\expandafter\def\csname PYG@tok@k\endcsname{\let\PYG@bf=\textbf\def\PYG@tc##1{\textcolor[rgb]{0.00,0.44,0.13}{##1}}}
\expandafter\def\csname PYG@tok@se\endcsname{\let\PYG@bf=\textbf\def\PYG@tc##1{\textcolor[rgb]{0.25,0.44,0.63}{##1}}}
\expandafter\def\csname PYG@tok@sd\endcsname{\let\PYG@it=\textit\def\PYG@tc##1{\textcolor[rgb]{0.25,0.44,0.63}{##1}}}

\def\PYGZbs{\char`\\}
\def\PYGZus{\char`\_}
\def\PYGZob{\char`\{}
\def\PYGZcb{\char`\}}
\def\PYGZca{\char`\^}
\def\PYGZam{\char`\&}
\def\PYGZlt{\char`\<}
\def\PYGZgt{\char`\>}
\def\PYGZsh{\char`\#}
\def\PYGZpc{\char`\%}
\def\PYGZdl{\char`\$}
\def\PYGZti{\char`\~}
% for compatibility with earlier versions
\def\PYGZat{@}
\def\PYGZlb{[}
\def\PYGZrb{]}
\makeatother

\begin{document}

\maketitle
\tableofcontents
\phantomsection\label{index::doc}


Contents:


\chapter{Introduction}
\label{intro:introduction}\label{intro:welcome-to-pybdf-s-documentation}\label{intro::doc}\begin{quote}\begin{description}
\item[{Author}] \leavevmode
Samuele Carcagno

\end{description}\end{quote}

pybdf is a python library to read BIOSEMI 24-bit BDF files.


\chapter{Download and Installation}
\label{intro:download-and-installation}

\section{Download}
\label{intro:download}
The latest release of pybdf can be downloaded from the python package index:

\href{http://pypi.python.org/pypi/pybdf/}{http://pypi.python.org/pypi/pybdf/}

For developers: the source code of pybdf is hosted on
github:

\href{https://github.com/sam81/pybdf}{https://github.com/sam81/pybdf}


\section{Installation}
\label{intro:installation}

\subsection{Requirements}
\label{intro:requirements}\begin{description}
\item[{To install pybdf you will need:}] \leavevmode\begin{itemize}
\item {} 
python

\item {} 
numpy

\item {} 
a fortran compiler

\end{itemize}

\end{description}

On all platforms, after having unpacked the archive
you can install pybdf by running:

\begin{Verbatim}[commandchars=\\\{\}]
python setup.py install
\end{Verbatim}

Note that pybdf has been built and tested only on Linux. I don't have
enhough time to build and test binaries for Windows and Mac OS X, but if
you try, please let me know how you get on with it.


\chapter{Usage}
\label{intro:usage}
To open a bdf file you need to create a bdfRecording
object as follows:

\begin{Verbatim}[commandchars=\\\{\}]
\PYG{n}{bdfRec} \PYG{o}{=} \PYG{n}{bdfRecording}\PYG{p}{(}\PYG{l+s}{'}\PYG{l+s}{res1.bdf}\PYG{l+s}{'}\PYG{p}{)}
\end{Verbatim}

you can then query the properties of the recording stored in the BDF header using the
appropriate functions, which are fully described {\hyperref[intro:module-label]{\emph{here}}}.
Some examples are shown below.

Get the duration of the recording:

\begin{Verbatim}[commandchars=\\\{\}]
\PYG{n}{bdfRec}\PYG{o}{.}\PYG{n}{recordDuration}
\end{Verbatim}

Get the sampling rate of each channel:

\begin{Verbatim}[commandchars=\\\{\}]
\PYG{n}{bdfRec}\PYG{o}{.}\PYG{n}{sampRate}
\end{Verbatim}

Get the channel labels:

\begin{Verbatim}[commandchars=\\\{\}]
\PYG{n}{bdfRec}\PYG{o}{.}\PYG{n}{chanLabels}
\end{Verbatim}

To read in the data use the following method:

\begin{Verbatim}[commandchars=\\\{\}]
\PYG{n}{rec} \PYG{o}{=} \PYG{n}{bdfRec}\PYG{o}{.}\PYG{n}{getData}\PYG{p}{(}\PYG{p}{)}
\end{Verbatim}

this returns a python dictionary
with the following fields:
\begin{itemize}
\item {} 
data : an array of floats with dimensions nChannels X nDataPoints

\item {} 
eventTable : a dictionary with the codes, indexes and durations of triggers

\item {} 
chanLabels : a list with the channel labels

\end{itemize}

For example, to get the value of the first sample of the recording,
in the first channel, you can type:

\begin{Verbatim}[commandchars=\\\{\}]
\PYG{n}{rec}\PYG{p}{[}\PYG{l+s}{'}\PYG{l+s}{data}\PYG{l+s}{'}\PYG{p}{]}\PYG{p}{[}\PYG{l+m+mi}{0}\PYG{p}{,}\PYG{l+m+mi}{0}\PYG{p}{]}
\end{Verbatim}

the same sample value, but for the second channel, is stored in:

\begin{Verbatim}[commandchars=\\\{\}]
\PYG{n}{rec}\PYG{p}{[}\PYG{l+s}{'}\PYG{l+s}{data}\PYG{l+s}{'}\PYG{p}{]}\PYG{p}{[}\PYG{l+m+mi}{1}\PYG{p}{,}\PYG{l+m+mi}{0}\PYG{p}{]}
\end{Verbatim}

The \emph{eventTable} contains a list of the trigger codes for the experimental conditions:

\begin{Verbatim}[commandchars=\\\{\}]
rec['eventTable{}`]['code']
\end{Verbatim}

as well as a list of the sample numbers (or indexes) at which they started in the recording, and a list of their durations in seconds:

\begin{Verbatim}[commandchars=\\\{\}]
rec['eventTable']['idx']
rec['eventTable{}`]['dur']
\end{Verbatim}

The ActiveTwo actually stores one trigger code for each recording sample rather than
a list of trigger onsets and durations as the \emph{eventTable} does. The ``raw'' trigger channel
with one trigger code for each recording sample can be retrieved by passing the argument \emph{trigChan = True}
to the \emph{getData()} function:

\begin{Verbatim}[commandchars=\\\{\}]
\PYG{n}{rec} \PYG{o}{=} \PYG{n}{bdfRec}\PYG{o}{.}\PYG{n}{getData}\PYG{p}{(}\PYG{n}{trigChan}\PYG{o}{=}\PYG{n+nb+bp}{True}\PYG{p}{)}
\end{Verbatim}

the ``raw'' trigger channel will then be returned in \emph{rec{[}'trigChan'{]}}.

It is also possible to retrieve additional system codes (bits 16-23 of the
status channel, see \href{http://www.biosemi.com/faq/trigger\_signals.htm}{http://www.biosemi.com/faq/trigger\_signals.htm}), like CMS
in/out-of range, battery low/OK. These are returned in \emph{rec{[}'sysCodeChan'{]}}
when the \emph{sysCodeChan = True} argument is passed to the \emph{getData()} function.
No particular effort has been made to decode these system codes.

Other usage examples are provided in the `examples' directory inside
the pybdf source archive.

Beware that pybdf does not check that you have sufficient RAM to
read all the data in a bdf file. If you try to read a file that is
too big for your hardware, you system may become slow or unresponsive.
Initially try reading only a small amount of data, and check how much
RAM that uses. You can read only a portion of the data by passing the
beginning and end arguments to the getData() function.
For example, to read the first 10 seconds of the recording, use:

\begin{Verbatim}[commandchars=\\\{\}]
\PYG{n}{rec} \PYG{o}{=} \PYG{n}{bdfRec}\PYG{o}{.}\PYG{n}{getData}\PYG{p}{(}\PYG{n}{beginning}\PYG{o}{=}\PYG{l+m+mi}{0}\PYG{p}{,} \PYG{n}{end}\PYG{o}{=}\PYG{l+m+mi}{10}\PYG{p}{)}
\end{Verbatim}


\chapter{Bugs}
\label{intro:bugs}
Please, report any bugs on github \href{https://github.com/sam81/pybdf/issues}{https://github.com/sam81/pybdf/issues}


\section{Known issues}
\label{intro:known-issues}
The filename or filepath of the BDF recording is currently limited
to 256 characters. That should be sufficient for most purposes.


\chapter{\texttt{pybdf} -- Class to read BIOSEMI BDF files}
\label{intro:module-label}\label{intro:pybdf-class-to-read-biosemi-bdf-files}

\chapter{Indices and tables}
\label{index:indices-and-tables}\begin{itemize}
\item {} 
\emph{genindex}

\item {} 
\emph{modindex}

\item {} 
\emph{search}

\end{itemize}



\renewcommand{\indexname}{Index}
\printindex
\end{document}
