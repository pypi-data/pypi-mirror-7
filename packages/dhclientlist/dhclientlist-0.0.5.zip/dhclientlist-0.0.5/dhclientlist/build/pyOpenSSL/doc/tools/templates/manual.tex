\documentclass{manual}

\title{Big Python Manual}

\author{Your Name Here}

% Please at least include a long-lived email address;
% the rest is at your discretion.
\authoraddress{
	Organization name, if applicable \\
	Street address, if you want to use it \\
	E-mail: \email{your-email@your.domain}
}

\date{April 30, 1999}		% update before release!
				% Use an explicit date so that reformatting
				% doesn't cause a new date to be used.  Setting
				% the date to \today can be used during draft
				% stages to make it easier to handle versions.

\release{x.y}			% release version; this is used to define the
				% \version macro

\makeindex			% tell \index to actually write the .idx file
\makemodindex			% If this contains a lot of module sections.


\begin{document}

\maketitle

% This makes the contents more accessible from the front page of the HTML.
\ifhtml
\chapter*{Front Matter\label{front}}
\fi

%Copyright \copyright{} 2001-2004 Python Software Foundation.
All rights reserved.

Copyright \copyright{} 2000 BeOpen.com.
All rights reserved.

Copyright \copyright{} 1995-2000 Corporation for National Research Initiatives.
All rights reserved.

Copyright \copyright{} 1991-1995 Stichting Mathematisch Centrum.
All rights reserved.

See the end of this document for complete license and permissions
information.


\begin{abstract}

\noindent
Big Python is a special version of Python for users who require larger 
keys on their keyboards.  It accomodates their special needs by ...

\end{abstract}

\tableofcontents


\chapter{...}

My chapter.


\appendix
\chapter{...}

My appendix.

The \code{\e appendix} markup need not be repeated for additional
appendices.


%
%  The ugly "%begin{latexonly}" pseudo-environments are really just to
%  keep LaTeX2HTML quiet during the \renewcommand{} macros; they're
%  not really valuable.
%
%  If you don't want the Module Index, you can remove all of this up
%  until the second \input line.
%
%begin{latexonly}
\renewcommand{\indexname}{Module Index}
%end{latexonly}
\input{mod\jobname.ind}		% Module Index

%begin{latexonly}
\renewcommand{\indexname}{Index}
%end{latexonly}
\input{\jobname.ind}			% Index

\end{document}
