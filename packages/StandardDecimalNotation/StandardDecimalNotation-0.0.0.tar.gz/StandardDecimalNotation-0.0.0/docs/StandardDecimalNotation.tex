% Generated by Sphinx.
\def\sphinxdocclass{report}
\documentclass[letterpaper,10pt,english]{sphinxmanual}
\usepackage[utf8]{inputenc}
\DeclareUnicodeCharacter{00A0}{\nobreakspace}
\usepackage[T1]{fontenc}
\usepackage{babel}
\usepackage{times}
\usepackage[Bjarne]{fncychap}
\usepackage{longtable}
\usepackage{sphinx}
\usepackage{multirow}


\title{Standard Decimal Notation Documentation}
\date{July 28, 2014}
\release{0.0.1}
\author{Shawn Pringle}
\newcommand{\sphinxlogo}{}
\renewcommand{\releasename}{Release}
\makeindex

\makeatletter
\def\PYG@reset{\let\PYG@it=\relax \let\PYG@bf=\relax%
    \let\PYG@ul=\relax \let\PYG@tc=\relax%
    \let\PYG@bc=\relax \let\PYG@ff=\relax}
\def\PYG@tok#1{\csname PYG@tok@#1\endcsname}
\def\PYG@toks#1+{\ifx\relax#1\empty\else%
    \PYG@tok{#1}\expandafter\PYG@toks\fi}
\def\PYG@do#1{\PYG@bc{\PYG@tc{\PYG@ul{%
    \PYG@it{\PYG@bf{\PYG@ff{#1}}}}}}}
\def\PYG#1#2{\PYG@reset\PYG@toks#1+\relax+\PYG@do{#2}}

\expandafter\def\csname PYG@tok@gd\endcsname{\def\PYG@tc##1{\textcolor[rgb]{0.63,0.00,0.00}{##1}}}
\expandafter\def\csname PYG@tok@gu\endcsname{\let\PYG@bf=\textbf\def\PYG@tc##1{\textcolor[rgb]{0.50,0.00,0.50}{##1}}}
\expandafter\def\csname PYG@tok@gt\endcsname{\def\PYG@tc##1{\textcolor[rgb]{0.00,0.27,0.87}{##1}}}
\expandafter\def\csname PYG@tok@gs\endcsname{\let\PYG@bf=\textbf}
\expandafter\def\csname PYG@tok@gr\endcsname{\def\PYG@tc##1{\textcolor[rgb]{1.00,0.00,0.00}{##1}}}
\expandafter\def\csname PYG@tok@cm\endcsname{\let\PYG@it=\textit\def\PYG@tc##1{\textcolor[rgb]{0.25,0.50,0.56}{##1}}}
\expandafter\def\csname PYG@tok@vg\endcsname{\def\PYG@tc##1{\textcolor[rgb]{0.73,0.38,0.84}{##1}}}
\expandafter\def\csname PYG@tok@m\endcsname{\def\PYG@tc##1{\textcolor[rgb]{0.13,0.50,0.31}{##1}}}
\expandafter\def\csname PYG@tok@mh\endcsname{\def\PYG@tc##1{\textcolor[rgb]{0.13,0.50,0.31}{##1}}}
\expandafter\def\csname PYG@tok@cs\endcsname{\def\PYG@tc##1{\textcolor[rgb]{0.25,0.50,0.56}{##1}}\def\PYG@bc##1{\setlength{\fboxsep}{0pt}\colorbox[rgb]{1.00,0.94,0.94}{\strut ##1}}}
\expandafter\def\csname PYG@tok@ge\endcsname{\let\PYG@it=\textit}
\expandafter\def\csname PYG@tok@vc\endcsname{\def\PYG@tc##1{\textcolor[rgb]{0.73,0.38,0.84}{##1}}}
\expandafter\def\csname PYG@tok@il\endcsname{\def\PYG@tc##1{\textcolor[rgb]{0.13,0.50,0.31}{##1}}}
\expandafter\def\csname PYG@tok@go\endcsname{\def\PYG@tc##1{\textcolor[rgb]{0.20,0.20,0.20}{##1}}}
\expandafter\def\csname PYG@tok@cp\endcsname{\def\PYG@tc##1{\textcolor[rgb]{0.00,0.44,0.13}{##1}}}
\expandafter\def\csname PYG@tok@gi\endcsname{\def\PYG@tc##1{\textcolor[rgb]{0.00,0.63,0.00}{##1}}}
\expandafter\def\csname PYG@tok@gh\endcsname{\let\PYG@bf=\textbf\def\PYG@tc##1{\textcolor[rgb]{0.00,0.00,0.50}{##1}}}
\expandafter\def\csname PYG@tok@ni\endcsname{\let\PYG@bf=\textbf\def\PYG@tc##1{\textcolor[rgb]{0.84,0.33,0.22}{##1}}}
\expandafter\def\csname PYG@tok@nl\endcsname{\let\PYG@bf=\textbf\def\PYG@tc##1{\textcolor[rgb]{0.00,0.13,0.44}{##1}}}
\expandafter\def\csname PYG@tok@nn\endcsname{\let\PYG@bf=\textbf\def\PYG@tc##1{\textcolor[rgb]{0.05,0.52,0.71}{##1}}}
\expandafter\def\csname PYG@tok@no\endcsname{\def\PYG@tc##1{\textcolor[rgb]{0.38,0.68,0.84}{##1}}}
\expandafter\def\csname PYG@tok@na\endcsname{\def\PYG@tc##1{\textcolor[rgb]{0.25,0.44,0.63}{##1}}}
\expandafter\def\csname PYG@tok@nb\endcsname{\def\PYG@tc##1{\textcolor[rgb]{0.00,0.44,0.13}{##1}}}
\expandafter\def\csname PYG@tok@nc\endcsname{\let\PYG@bf=\textbf\def\PYG@tc##1{\textcolor[rgb]{0.05,0.52,0.71}{##1}}}
\expandafter\def\csname PYG@tok@nd\endcsname{\let\PYG@bf=\textbf\def\PYG@tc##1{\textcolor[rgb]{0.33,0.33,0.33}{##1}}}
\expandafter\def\csname PYG@tok@ne\endcsname{\def\PYG@tc##1{\textcolor[rgb]{0.00,0.44,0.13}{##1}}}
\expandafter\def\csname PYG@tok@nf\endcsname{\def\PYG@tc##1{\textcolor[rgb]{0.02,0.16,0.49}{##1}}}
\expandafter\def\csname PYG@tok@si\endcsname{\let\PYG@it=\textit\def\PYG@tc##1{\textcolor[rgb]{0.44,0.63,0.82}{##1}}}
\expandafter\def\csname PYG@tok@s2\endcsname{\def\PYG@tc##1{\textcolor[rgb]{0.25,0.44,0.63}{##1}}}
\expandafter\def\csname PYG@tok@vi\endcsname{\def\PYG@tc##1{\textcolor[rgb]{0.73,0.38,0.84}{##1}}}
\expandafter\def\csname PYG@tok@nt\endcsname{\let\PYG@bf=\textbf\def\PYG@tc##1{\textcolor[rgb]{0.02,0.16,0.45}{##1}}}
\expandafter\def\csname PYG@tok@nv\endcsname{\def\PYG@tc##1{\textcolor[rgb]{0.73,0.38,0.84}{##1}}}
\expandafter\def\csname PYG@tok@s1\endcsname{\def\PYG@tc##1{\textcolor[rgb]{0.25,0.44,0.63}{##1}}}
\expandafter\def\csname PYG@tok@gp\endcsname{\let\PYG@bf=\textbf\def\PYG@tc##1{\textcolor[rgb]{0.78,0.36,0.04}{##1}}}
\expandafter\def\csname PYG@tok@sh\endcsname{\def\PYG@tc##1{\textcolor[rgb]{0.25,0.44,0.63}{##1}}}
\expandafter\def\csname PYG@tok@ow\endcsname{\let\PYG@bf=\textbf\def\PYG@tc##1{\textcolor[rgb]{0.00,0.44,0.13}{##1}}}
\expandafter\def\csname PYG@tok@sx\endcsname{\def\PYG@tc##1{\textcolor[rgb]{0.78,0.36,0.04}{##1}}}
\expandafter\def\csname PYG@tok@bp\endcsname{\def\PYG@tc##1{\textcolor[rgb]{0.00,0.44,0.13}{##1}}}
\expandafter\def\csname PYG@tok@c1\endcsname{\let\PYG@it=\textit\def\PYG@tc##1{\textcolor[rgb]{0.25,0.50,0.56}{##1}}}
\expandafter\def\csname PYG@tok@kc\endcsname{\let\PYG@bf=\textbf\def\PYG@tc##1{\textcolor[rgb]{0.00,0.44,0.13}{##1}}}
\expandafter\def\csname PYG@tok@c\endcsname{\let\PYG@it=\textit\def\PYG@tc##1{\textcolor[rgb]{0.25,0.50,0.56}{##1}}}
\expandafter\def\csname PYG@tok@mf\endcsname{\def\PYG@tc##1{\textcolor[rgb]{0.13,0.50,0.31}{##1}}}
\expandafter\def\csname PYG@tok@err\endcsname{\def\PYG@bc##1{\setlength{\fboxsep}{0pt}\fcolorbox[rgb]{1.00,0.00,0.00}{1,1,1}{\strut ##1}}}
\expandafter\def\csname PYG@tok@kd\endcsname{\let\PYG@bf=\textbf\def\PYG@tc##1{\textcolor[rgb]{0.00,0.44,0.13}{##1}}}
\expandafter\def\csname PYG@tok@ss\endcsname{\def\PYG@tc##1{\textcolor[rgb]{0.32,0.47,0.09}{##1}}}
\expandafter\def\csname PYG@tok@sr\endcsname{\def\PYG@tc##1{\textcolor[rgb]{0.14,0.33,0.53}{##1}}}
\expandafter\def\csname PYG@tok@mo\endcsname{\def\PYG@tc##1{\textcolor[rgb]{0.13,0.50,0.31}{##1}}}
\expandafter\def\csname PYG@tok@mi\endcsname{\def\PYG@tc##1{\textcolor[rgb]{0.13,0.50,0.31}{##1}}}
\expandafter\def\csname PYG@tok@kn\endcsname{\let\PYG@bf=\textbf\def\PYG@tc##1{\textcolor[rgb]{0.00,0.44,0.13}{##1}}}
\expandafter\def\csname PYG@tok@o\endcsname{\def\PYG@tc##1{\textcolor[rgb]{0.40,0.40,0.40}{##1}}}
\expandafter\def\csname PYG@tok@kr\endcsname{\let\PYG@bf=\textbf\def\PYG@tc##1{\textcolor[rgb]{0.00,0.44,0.13}{##1}}}
\expandafter\def\csname PYG@tok@s\endcsname{\def\PYG@tc##1{\textcolor[rgb]{0.25,0.44,0.63}{##1}}}
\expandafter\def\csname PYG@tok@kp\endcsname{\def\PYG@tc##1{\textcolor[rgb]{0.00,0.44,0.13}{##1}}}
\expandafter\def\csname PYG@tok@w\endcsname{\def\PYG@tc##1{\textcolor[rgb]{0.73,0.73,0.73}{##1}}}
\expandafter\def\csname PYG@tok@kt\endcsname{\def\PYG@tc##1{\textcolor[rgb]{0.56,0.13,0.00}{##1}}}
\expandafter\def\csname PYG@tok@sc\endcsname{\def\PYG@tc##1{\textcolor[rgb]{0.25,0.44,0.63}{##1}}}
\expandafter\def\csname PYG@tok@sb\endcsname{\def\PYG@tc##1{\textcolor[rgb]{0.25,0.44,0.63}{##1}}}
\expandafter\def\csname PYG@tok@k\endcsname{\let\PYG@bf=\textbf\def\PYG@tc##1{\textcolor[rgb]{0.00,0.44,0.13}{##1}}}
\expandafter\def\csname PYG@tok@se\endcsname{\let\PYG@bf=\textbf\def\PYG@tc##1{\textcolor[rgb]{0.25,0.44,0.63}{##1}}}
\expandafter\def\csname PYG@tok@sd\endcsname{\let\PYG@it=\textit\def\PYG@tc##1{\textcolor[rgb]{0.25,0.44,0.63}{##1}}}

\def\PYGZbs{\char`\\}
\def\PYGZus{\char`\_}
\def\PYGZob{\char`\{}
\def\PYGZcb{\char`\}}
\def\PYGZca{\char`\^}
\def\PYGZam{\char`\&}
\def\PYGZlt{\char`\<}
\def\PYGZgt{\char`\>}
\def\PYGZsh{\char`\#}
\def\PYGZpc{\char`\%}
\def\PYGZdl{\char`\$}
\def\PYGZhy{\char`\-}
\def\PYGZsq{\char`\'}
\def\PYGZdq{\char`\"}
\def\PYGZti{\char`\~}
% for compatibility with earlier versions
\def\PYGZat{@}
\def\PYGZlb{[}
\def\PYGZrb{]}
\makeatother

\begin{document}

\maketitle
\tableofcontents
\phantomsection\label{index::doc}


Contents:
\phantomsection\label{index:module-qsdn}\index{qsdn (module)}\begin{description}
\item[{... module:: qsdn}] \leavevmode\begin{quote}\begin{description}
\item[{platform}] \leavevmode
Unix, Windows

\item[{synopsis}] \leavevmode
This module allows for parsing, validation and production of numeric literals, written with thousand separators through out the number.  Often underlying system libraries for working with locales neglect to put thousand separators (commas) after the decimal place or they sometimes use scientific notation.  The classes inherit from the Qt classes for making things less complex.

\end{description}\end{quote}

Thousand separators in general will not always be commas but instead will be different according to the locale settings.  In Windows for example, the user can set his thousand separator to any character.  Support for converting strings directly to Decimals and from Decimals to strings is included.

Also, numbers are always expressed in standard decimal notation.

Care has been taken to overload all of the members in a way
that is consistent with the base class QLocale and QValidator.

\end{description}
\index{QSDNLocale (class in qsdn)}

\begin{fulllineitems}
\phantomsection\label{index:qsdn.QSDNLocale}\pysiglinewithargsret{\strong{class }\code{qsdn.}\bfcode{QSDNLocale}}{\emph{\_name=None}, \emph{p\_mandatory\_decimals=Decimal(`0')}, \emph{p\_maximum\_decimals=Decimal(`Infinity')}}{}~\begin{description}
\item[{For a QSDNlocale, locale:  }] \leavevmode
To get the Decimal of a string, s, use:

(d, ok) = locale.toDecimal(s, 10)

The value d is your decimal, and you should check ok before you trust d.

To get the string representation use:

s = locale.toString(d)

\end{description}
\index{c() (qsdn.QSDNLocale static method)}

\begin{fulllineitems}
\phantomsection\label{index:qsdn.QSDNLocale.c}\pysiglinewithargsret{\strong{static }\bfcode{c}}{}{}
Returns the C locale.  In the C locale, to* routines will not accept group separtors and do not produce them.

\end{fulllineitems}

\index{system() (qsdn.QSDNLocale static method)}

\begin{fulllineitems}
\phantomsection\label{index:qsdn.QSDNLocale.system}\pysiglinewithargsret{\strong{static }\bfcode{system}}{}{}
Returns the system default for QSDNLocale.

\end{fulllineitems}

\index{toDecimal() (qsdn.QSDNLocale method)}

\begin{fulllineitems}
\phantomsection\label{index:qsdn.QSDNLocale.toDecimal}\pysiglinewithargsret{\bfcode{toDecimal}}{\emph{s}, \emph{base=0}}{}
This creates a decimal representation of s.

It returns an ordered pair.  The first of the pair is the Decimal number, the second of the pair indicates whether the string had a valid representation of that number.  You should always check the second of the ordered pair before using the decimal returned.
\begin{quote}\begin{description}
\item[{Note }] \leavevmode
Make sure you use 10 as the second argument or it may interpret the string as octal!

\end{description}\end{quote}

Like the other to* functions of QLocale as well as this class QSDNLocale, interpret a 
a string and parse it and return a Decimal.  The base value is used to determine what base to use.

If base is not set, numbers such as `0777' will be interpreted as octal.  The string `0x33' will
be interpreted as hexadecimal and `777' will be interpreted as a decimal.  It is done this way
so this works like toLong, toInt, toFloat, etc...
Leading and trailing whitespace is ignored.

\end{fulllineitems}

\index{toDouble() (qsdn.QSDNLocale method)}

\begin{fulllineitems}
\phantomsection\label{index:qsdn.QSDNLocale.toDouble}\pysiglinewithargsret{\bfcode{toDouble}}{\emph{s}}{}
This creates a floating point representation of s.

It returns an ordered pair.  The first of the pair is the number, the second of the pair indicates whether the string had a valid representation of that number.  You should always check the second of the ordered pair before using the number returned.

\end{fulllineitems}

\index{toFloat() (qsdn.QSDNLocale method)}

\begin{fulllineitems}
\phantomsection\label{index:qsdn.QSDNLocale.toFloat}\pysiglinewithargsret{\bfcode{toFloat}}{\emph{s}}{}
This creates a floating point representation of s.

It returns an ordered pair.  The first of the pair is the number, the second of the pair indicates whether the string had a valid representation of that number.  You should always check the second of the ordered pair before using the number returned.

\end{fulllineitems}

\index{toLongLong() (qsdn.QSDNLocale method)}

\begin{fulllineitems}
\phantomsection\label{index:qsdn.QSDNLocale.toLongLong}\pysiglinewithargsret{\bfcode{toLongLong}}{\emph{s}, \emph{base=0}}{}
This creates a numeric representation of s.

It returns an ordered pair.  The first of the pair is the number, the second of the pair indicates whether the string had a valid representation of that number.  You should always check the second of the ordered pair before using the number returned.
\begin{quote}
\begin{quote}\begin{description}
\item[{note}] \leavevmode
Make sure you use 10 as the second argument or it may interpret the string as octal!

\end{description}\end{quote}
\end{quote}

If base is not set, numbers such as `0777' will be interpreted as octal.  The string `0x33' will
be interpreted as hexadecimal and `777' will be interpreted as a decimal.  It is done this way
so this works like toLong, toInt, toFloat, etc...

Leading and trailing whitespace is ignored.

\end{fulllineitems}

\index{toShort() (qsdn.QSDNLocale method)}

\begin{fulllineitems}
\phantomsection\label{index:qsdn.QSDNLocale.toShort}\pysiglinewithargsret{\bfcode{toShort}}{\emph{s}, \emph{base=0}}{}
This creates a numeric representation of s.

It returns an ordered pair.  The first of the pair is the number, the second of the pair indicates whether the string had a valid representation of that number.  You should always check the second of the ordered pair before using the number returned.
\begin{quote}
\begin{quote}\begin{description}
\item[{note}] \leavevmode
Make sure you use 10 as the second argument or it may interpret the string as octal!

\end{description}\end{quote}
\end{quote}

If base is not set, numbers such as `0777' will be interpreted as octal.  The string `0x33' will
be interpreted as hexadecimal and `777' will be interpreted as a decimal.  It is done this way
so this works like toLong, toInt, toFloat, etc...

Leading and trailing whitespace is ignored.

\end{fulllineitems}

\index{toString() (qsdn.QSDNLocale method)}

\begin{fulllineitems}
\phantomsection\label{index:qsdn.QSDNLocale.toString}\pysiglinewithargsret{\bfcode{toString}}{\emph{x}, \emph{arg2=None}, \emph{arg3=None}}{}
Convert any given Decimal, double, Date, Time, int or long to a string.

Numbers are always converted to Standard decimal notation.  That is to say,
numbers are never converted to scientifc notation.

The way toString is controlled:
If passing a decimal.Decimal typed value, the precision is recorded in the 
number itself.  So, D(`4.00') will be expressed as `4.00' and not `4'.
D(`4') will be expressed as `4'.

When a number passed is NOT a Decimal, numbers are created in the following way:
Two extra parameters, set during creation of the locale, determines how 
many digits will appear in the result of toString().
For example, we have a number like 5.1 and mandatory decimals was set    
to 2, toString(5.1) should return `5.10'.  A number like 6 would be `6.00'.
A number like 5.104 would depend on the maximum decimals setting, also 
set at construction of the locale:
\_maximum\_decimals controls the maximum number of decimals after the decimal point
So, if \_maximum\_decimals is 6 and \_mandatory\_decimals is 2 then 
toString(Decimal(`3.1415929')) is `3.141,592'.
Notice the number is truncated and not rounded.  
Consider rounding a copy of the number before displaying.

\end{fulllineitems}

\index{toUInt() (qsdn.QSDNLocale method)}

\begin{fulllineitems}
\phantomsection\label{index:qsdn.QSDNLocale.toUInt}\pysiglinewithargsret{\bfcode{toUInt}}{\emph{s}, \emph{base=0}}{}
This creates a numeric representation of s.

It returns an ordered pair.  The first of the pair is the number, the second of the pair indicates whether the string had a valid representation of that number.  You should always check the second of the ordered pair before using the number returned.
\begin{quote}
\begin{quote}\begin{description}
\item[{note}] \leavevmode
Make sure you use 10 as the second argument or it may interpret the string as octal!

\end{description}\end{quote}
\end{quote}

If base is not set, numbers such as `0777' will be interpreted as octal.  The string `0x33' will
be interpreted as hexadecimal and `777' will be interpreted as a decimal.  It is done this way
so this works like toLong, toInt, toFloat, etc...

Leading and trailing whitespace is ignored.

\end{fulllineitems}

\index{toULongLong() (qsdn.QSDNLocale method)}

\begin{fulllineitems}
\phantomsection\label{index:qsdn.QSDNLocale.toULongLong}\pysiglinewithargsret{\bfcode{toULongLong}}{\emph{s}, \emph{base=0}}{}
This creates a numeric representation of s.

It returns an ordered pair.  The first of the pair is the number, the second of the pair indicates whether the string had a valid representation of that number.  You should always check the second of the ordered pair before using the number returned.
\begin{quote}
\begin{quote}\begin{description}
\item[{note}] \leavevmode
Make sure you use 10 as the second argument or it may interpret the string as octal!

\end{description}\end{quote}
\end{quote}

If base is not set, numbers such as `0777' will be interpreted as octal.  The string `0x33' will
be interpreted as hexadecimal and `777' will be interpreted as a decimal.  It is done this way
so this works like toLong, toInt, toFloat, etc...

Leading and trailing whitespace is ignored.

\end{fulllineitems}

\index{toUShort() (qsdn.QSDNLocale method)}

\begin{fulllineitems}
\phantomsection\label{index:qsdn.QSDNLocale.toUShort}\pysiglinewithargsret{\bfcode{toUShort}}{\emph{s}, \emph{base=0}}{}
This creates a numeric representation of s.

It returns an ordered pair.  The first of the pair is the number, the second of the pair indicates whether the string had a valid representation of that number.  You should always check the second of the ordered pair before using the number returned.
\begin{quote}
\begin{quote}\begin{description}
\item[{note}] \leavevmode
Make sure you use 10 as the second argument or it may interpret the string as octal!

\end{description}\end{quote}
\end{quote}

If base is not set, numbers such as `0777' will be interpreted as octal.  The string `0x33' will
be interpreted as hexadecimal and `777' will be interpreted as a decimal.  It is done this way
so this works like toLong, toInt, toFloat, etc...

Leading and trailing whitespace is ignored.

\end{fulllineitems}


\end{fulllineitems}

\index{QSDNNumericValidator (class in qsdn)}

\begin{fulllineitems}
\phantomsection\label{index:qsdn.QSDNNumericValidator}\pysiglinewithargsret{\strong{class }\code{qsdn.}\bfcode{QSDNNumericValidator}}{\emph{maximum\_decamals=1000}, \emph{maximum\_decimals=1000}, \emph{use\_space=False}, \emph{parent=None}}{}
QSDNNumericValidator limits the number of digits after the decimal
point and the number of digits before.
\begin{quote}

bitcoin                         :  QSDNNumericValidator(8, 8)
US dollars less than \$1,000,000 :  QSDNNumericValidator(6, 2)

If use space is true, spaces are added on the left such that the location
of decimal point remains constant.  Numbers like `10,000.004', `102.126' become 
aligned.
Bitcoin amounts:
\begin{quote}

`        0.004,3'
`       10.4'
`      320.0'
`        0.000,004'
\end{quote}
\begin{description}
\item[{U.S. dollar amounts;  }] \leavevmode
dollar = QSDNNumericValidator(6,2)
s = `42.1'
dollar.validate(s=`42.1', 2)   =\textgreater{}  s = `     42.10'
s=`50000'
dollar.toString(s)              =\textgreater{} s = ` 50,000.00'

\end{description}
\end{quote}
\index{decamals() (qsdn.QSDNNumericValidator method)}

\begin{fulllineitems}
\phantomsection\label{index:qsdn.QSDNNumericValidator.decamals}\pysiglinewithargsret{\bfcode{decamals}}{}{}
gets the number of decimal points that are allowed \textbf{before} the decimal point

\end{fulllineitems}

\index{decimals() (qsdn.QSDNNumericValidator method)}

\begin{fulllineitems}
\phantomsection\label{index:qsdn.QSDNNumericValidator.decimals}\pysiglinewithargsret{\bfcode{decimals}}{}{}
gets the number of decimal points that are allowed \emph{after} the decimal point

\end{fulllineitems}

\index{locale() (qsdn.QSDNNumericValidator method)}

\begin{fulllineitems}
\phantomsection\label{index:qsdn.QSDNNumericValidator.locale}\pysiglinewithargsret{\bfcode{locale}}{}{}
get the locale used by this validator

\end{fulllineitems}

\index{setDecamals() (qsdn.QSDNNumericValidator method)}

\begin{fulllineitems}
\phantomsection\label{index:qsdn.QSDNNumericValidator.setDecamals}\pysiglinewithargsret{\bfcode{setDecamals}}{\emph{i}}{}
sets the number of decimal digits that should be allowed \textbf{before} the decimal point

\end{fulllineitems}

\index{setDecimals() (qsdn.QSDNNumericValidator method)}

\begin{fulllineitems}
\phantomsection\label{index:qsdn.QSDNNumericValidator.setDecimals}\pysiglinewithargsret{\bfcode{setDecimals}}{\emph{i}}{}
sets the number of decimal digits that should be allowed \textbf{after} the decimal point

\end{fulllineitems}

\index{setLocale() (qsdn.QSDNNumericValidator method)}

\begin{fulllineitems}
\phantomsection\label{index:qsdn.QSDNNumericValidator.setLocale}\pysiglinewithargsret{\bfcode{setLocale}}{\emph{plocale}}{}
Set the locale used by this Validator.

\end{fulllineitems}

\index{validate() (qsdn.QSDNNumericValidator method)}

\begin{fulllineitems}
\phantomsection\label{index:qsdn.QSDNNumericValidator.validate}\pysiglinewithargsret{\bfcode{validate}}{\emph{s}, \emph{pos}}{}
Validates s, by adjusting the position of the commas to be in the correct places and adjusting pos accordingly as well as space in order to keep decimal points aligned when varying sized numbers are put one above the other.

\end{fulllineitems}


\end{fulllineitems}



\chapter{Indices and tables}
\label{index:welcome-to-standard-decimal-notation-s-documentation}\label{index:indices-and-tables}\begin{itemize}
\item {} 
\emph{genindex}

\item {} 
\emph{modindex}

\item {} 
\emph{search}

\end{itemize}


\renewcommand{\indexname}{Python Module Index}
\begin{theindex}
\def\bigletter#1{{\Large\sffamily#1}\nopagebreak\vspace{1mm}}
\bigletter{q}
\item {\texttt{qsdn}}, \pageref{index:module-qsdn}
\end{theindex}

\renewcommand{\indexname}{Index}
\printindex
\end{document}
