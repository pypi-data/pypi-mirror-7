\documentclass{article}
\usepackage{amsmath}
\usepackage{amssymb}
\usepackage{amsthm}
\usepackage{hyperref}

\usepackage[margin=1in]{geometry}
\newtheorem{thm}{Theorem}[section]
\newtheorem{prop}{Proposition}[thm]
\newtheorem{lemma}{Lemma}[thm]
\newtheorem{defn}[thm]{Definition}
\newtheorem{cor}{Corollary}[thm]

\newcommand{\p}[1]{P^{(#1)}}
\newcommand{\x}{{\bf X}}
\newcommand{\set}[1]{\left\{#1\right\}}

\begin{document}

\section{git+latex workflow}

The first step in efficiently managing a git+latex workflow is to
make a few changes to your \LaTeX habits. \\

For starters, write each sentence on a separate line. Git was written
to version control source code, where each line is distinct and has a
specific purpose. When you write documents in \LaTeX, you often think
in terms of paragraphs and write it as a free flowing document.
However, in git, changes to a single word in a paragraph get recorded
as a change to the entire paragraph. \\

If you're writing a long document in latex, I'd suggest splitting
different chapters into their own files and call them in the main
file using the \emph{include} command. This way it is easier for you
to edit a localized part of your work, and it is also easier for
version control, as you know what changes have been made to each
chapter, instead of having to figure it out from the logs of one big
file. \\

\url{http://stackoverflow.com/a/6190412/788553}


\end{document}
