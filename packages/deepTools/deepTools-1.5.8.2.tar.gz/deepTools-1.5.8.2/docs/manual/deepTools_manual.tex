\documentclass[11pt,a4paper]{article}

\usepackage{url}
\usepackage[utf8]{inputenc}

\begin{document}
\title{DeepTools: user-friendly tools for the normalization and
  visualization of deep-sequencing data} 

\author{Fidel Ramírez, Friederike Dündar, Sarah Diehl, Björn Grüning}

\date{\today}

\maketitle


\tableofcontents

\section{Tools for quality control}

\subsection{bamFingerprint}

This tool is based on a method developed by Diaz et al. (2012). Stat
Appl Genet Mol Biol 11(3).  The resulting plot can be used to assess
the strength of a ChIP (for factors that bind to narrow regions).  The
tool first samples indexed BAM files and counts all reads overlapping
a window (bin) of specified length.  These counts are then sorted
according to their rank and the cumulative sum of read counts are
plotted. An ideal input with perfect uniform distribution of reads
along the genome (i.e. without enrichments in open chromatin etc.)
should generate a straight diagonal line. A very specific and strong
ChIP enrichment will be indicated by a prominent and steep rise of the
cumulative sum towards the highest rank. This means that a big chunk
of reads from the ChIP sample is located in few bins which corresponds
to high, narrow enrichments seen for transcription factors.


\subsection{computeGCBias}

This tool computes the GC bias using the method proposed by Benjamini
and Speed (2012). Nucleic Acids Res. (see below for more explanations)
The output is used to plot the bias and can also be used later on to
correct the bias with the tool correctGCbias.  There are two plots
produced by the tool: a boxplot showing the absolute read numbers per
genomic-GC bin and an x-y plot depicting the ratio of
observed/expected reads per genomic GC content bin.


\textbf{Summary of the method used}

In order to estimate how many reads with what kind of GC content one
should have sequenced, we first need to determine how many regions the
specific reference genome contains for each amount of GC content,
i.e. how many regions in the genome have 50\% GC (or 10\% GC or 90\%
GC or...).  We then sample a large number of equally sized genome bins
and count how many times we see a bin with 50\% GC (or 10\% GC or 90\%
or...). These EXPECTED values are independent of any  sequencing as it
only depends on the respective reference genome (i.e. it will most
likely vary between mouse and fruit fly due to their genome's
different GC contents).  The OBSERVED values are based on the reads
from the sequenced sample. Instead of noting how many genomic regions
there are per GC content, we now count the reads per GC content.  In
an ideal sample without GC bias, the ratio of OBSERVED/EXPECTED values
should be close to 1 regardless of the GC content. Due to PCR
(over)amplifications, the majority of ChIP samples usually shows a
significant bias towards reads with high GC content (>50\%)


\section{Tools for normalization}

\section{correctGCBias}

This tool requires the output from computeGCBias to correct the given
BAM files according to the method proposed by Benjamini and Speed
(2012). Nucleic Acids Res.  The resulting BAM files can be used in any
downstream analyses, but be aware that you should not filter out
duplicates from here on.


\subsection{bigwigCompare}

This tool compares two bigwig files based on the number of mapped reads. To
compare the bigwig files the genome is partitioned into bins of equal size,
then the number of reads found in each BAM file are counted for such bins and
finally a summarizing value is reported. This value can be the ratio of the
number of reads per bin, the log2 of the ratio, the sum or the difference.


\subsection{bamCoverage}

Given a BAM file, this tool generates a bigWig or bedGraph file of
fragment or read coverages. The way the method works is by first
calculating all the number of reads (either extended to match the
fragment length or not) that overlap each bin in the genome. Bins with
zero counts are skipped, i.e. not added to the output file. The
resulting read counts can be normalized using either a given scaling
factor, the RPKM formula or to get a 1x depth of coverage (RPGC).


\subsection{bamCorrelate}

This tool is useful to assess the overall similarity of different BAM
files. A typical application is to check the correlation between
replicates or published data sets.

The tool splits the genomes into bins of given length. For each bin,
the number of reads found in each BAM file is counted and a
correlation is computed for all pairs of BAM files.


\subsection{bamCompare}

This tool compares two BAM files based on the number of mapped
reads. To compare the BAM files, the genome is partitioned into bins
of equal size, the reads are counted for each bin and each BAM file
and finally, a summarizing value is reported.  This value can be the
ratio of the number of reads per bin, the log2 of the ratio or the
difference.  This tool can normalize the number of reads on each BAM
file using the SES method proposed by Diaz et al.  (2012). Stat Appl
Genet Mol Biol 11(3). Normalization based on read counts is also
available. The output is either a bedGraph or a bigWig file containing
the bin location and the resulting comparison values.  If paired-end
reads are present, the fragment length reported in the BAM file is
used by default.


\section{Tools for visualization}

\subsection{computeMatrix}

This tool summarizes and prepares an intermediary file containing
scores associated with genomic regions that can be used afterwards to
plot a heatmap or a profile. Typically, these genomic regions are
genes, but any other regions defined in a BED or INTERVAL format can
be used. This tool can also be used to filter and sort regions
according to their score.


\subsection{Profiler}

This tool creates a profile plot for a score associated to genomic regions.
Typically, these regions are genes, but any other regions defined in a BED or
INTERVAL format will work. A preprocessed matrix generated by the tool
computeMatrix is required.


\subsection{Heatmapper}

The heatmapper visualizes scores associated with genomic regions, for
example ChIP enrichment values around the TSS of genes. Those values
can be visualized individually along each of the regions provided by
the user in INTERVAL or BED format. In addition to the heatmap, an
average profile plot is plotted on top of the heatmap (can be turned
off by the user; it can also be generated separately by the tool
profiler). We implemented vast optional parameters and we encourage
you to play around with the min/max values displayed in the heatmap as
well as with the different coloring options. If you would like to plot
heatmaps for different groups of genomic regions individually,
e.g. one plot per chromosome, simply supply each group as an
individual BED file.





\section{Credits}

%% we need to add the grant numbers 

If you would like to give us feedback or you run into any trouble,
please send an email to deeptools@googlegroups.com

This tool is developed by the 
\url{Bioinformatics and Deep-Sequencing Unit}{http://www3.ie-freiburg.mpg.de/facilities/research-facilities/bioinformatics-and-deep-sequencing-unit/}
at the 
\url{Max Planck Institute for Immunobiology and Epigenetics}{http://www3.ie-freiburg.mpg.de}.

\end{document}
